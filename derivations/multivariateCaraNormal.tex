\documentclass[pdftex,12pt,a4paper]{article}


% use user packages
\usepackage[margin=1in]{geometry}  % set the margins to 1in on all sides
\usepackage{graphicx}              % to include figures
\usepackage{amsmath}               % great math stuff
\usepackage{amsfonts}              % for blackboard bold, etc
\usepackage{amsthm}                % better theorem environments
\usepackage{subfigure}
\usepackage{hyperref, url}

% use theorems
\newtheorem{theorem}{Theorem}[section]
\newtheorem{lemma}[theorem]{Lemma}
\newtheorem{proposition}[theorem]{Proposition}
\newtheorem{cor}[theorem]{Corollary}
\newtheorem{conj}[theorem]{Conjecture}
\newtheorem{definition}{Definition}
\newtheorem{remark}{Remark}

\newcommand{\comment}[1]{\textbf{[@J: #1]}}

\begin{document}

\title{Projecting a multivariate SDF onto each stock's return space: analytical derivations}
\author{J Li}
\date{\today}
\maketitle

\abstract

This is the key step of making joint estimation of stock-specific risk adjustments feasible. Currently it is a one-period model, and also has strong normality assumptions. Later may want to use multi-period (and then lead to one-period), and potentially use a mixture of normals. Also need some justification of the type of utility I'm using. 

\section{Primitives}

Set up: 

\begin{itemize}
\item We assume that the relevant state variables in the economy are a list of $k$ factor returns $F \sim \mathcal{N}(\mu_F, \Sigma_F)$. For instance, $F$ can be the standard Fama-French 3 factors plus Momentum. 
\item Return of each individual stock follows a factor structure: $R = \beta' F + e$, where $\beta$ is a $k$-by-1 vector, $e \sim \mathcal{N}(0, \sigma^2_e)$ and independent of $F$. The normality assumptions here are indispensable to yield analytical solutions. 
\item We assume that agents have a multivariate CARA type utility function that depends on the factors. Therefore the SDF $m(F) = \delta \exp(-\gamma' F)$, where $\delta < 1$ is the subjective impatience parameter, and $\gamma = (\gamma_1, ..., \gamma_k)'$ reflects risk preferences. \comment{Need justification... references about multivariate preferences of this kind? ICAPM?}
\end{itemize}

Now, we need to project $m(F)$ onto the return space of the single-name stock. In particular, we need: 

\begin{equation}
m(r) = \mathbb{E}_F(m(F) | R = r)
\end{equation}

This, due to the nice normality and CARA assumptions, have an analytical solution. 

\section{Derivation}

Let's denote the posterior variable $\tilde{F} \sim F | R = r$, and denote expectation under the posterior distribution as $\tilde{\mathbb{E}}(\cdot)$. $\tilde{\mu}_F(\beta, r)$ and $\tilde{\Sigma}_F(\beta, r)$ denote the posterior mean and covariances of $F$ given $R = r$. In this section, we first derive the distribution of $\tilde{F}$ by exploiting conjugate prior conditions, and then work out $m(r)$ by using a moment-generating-function formula. 

\subsection{Posterior Distribution: $\tilde{F}$}

This derivation follows the case of portfolio-view under Black-Litterman updating. The math behind can be justified using either Theil's mixed-estimation, or a Bayesian argument. Jay Walter's ``The Black-Litterman Model In Detail'' has good references. On a high level, think about $F$ as unobserved state variables, and $R = r$ as a noisy, incomplete observation that gives some updating information about $F$. 

Cutting to the chase, $\tilde{F}$ is a multivariate normal, and the posterior mean and covariance of $F$ given observing $R = r$ is:

\begin{enumerate}
\item 
\begin{align}
\tilde{\mu}_F(\beta, r) & = (\Sigma_F^{-1} + \frac{\beta \beta'}{\sigma^2_e})^{-1} (\Sigma_F \mu_F + \frac{\beta}{\sigma^2_e} r) \\
& = \underbrace{(\Sigma_F^{-1} + \frac{\beta \beta'}{\sigma^2_e})^{-1} \Sigma_F \mu_F}_{\mu_a(\beta)} + \underbrace{(\Sigma_F^{-1} + \frac{\beta \beta'}{\sigma^2_e})^{-1} \frac{\beta}{\sigma^2_e}}_{\mu_b(\beta)} r \label{posteriorMean}
\end{align}

\item 
\begin{equation}
\tilde{\Sigma}_F(\beta, r) = \tilde{\Sigma}_F(\beta) = (\Sigma_F^{-1} + \frac{\beta \beta'}{\sigma^2_e})^{-1} \label{posteriorCovariance}
\end{equation}
\end{enumerate}

\textbf{Some remarks:}

\begin{itemize}
\item $\tilde{\mu}_F$ is a precision-weighted-average of prior mean $\mu_F$ and noisy observations $r$. It is linear in $r$: $\tilde{\mu}_F(\beta,r) = \mu_a(\beta) + \mu_b(\beta) r$. At the same time, $\tilde{\Sigma}_F$ does not depend on $r$. 
\item As expected, $\tilde{\Sigma}_F \le \Sigma_F$, in the sense that $\Sigma - \tilde{\Sigma}_F$ is p.d., as $\beta \beta' \ge 0$. Equality holds only if $\beta = 0$, in which case $R$ is pure noise does not tell us anything about $F$. 
\item $r$'s impact on $\tilde{\mu}_F$ increases in the absolute magnitude of $\beta$ \footnote{Make more formal.} and decreases in $\sigma^2_e$, as expected. $\beta$ governs signal while $\sigma_e^2$ governs noise. When $\beta = 0$, $\tilde{\mu}_F \equiv \mu_F$. 
\end{itemize}

We will write out the math steps. This stuff should go into an appendix. 

\subsection{Projected Discount Factor}

Now we are ready to project the SDF. Luckily, we can use the formula of moment generating function for multivariate normal (\url{http://en.wikipedia.org/wiki/Moment-generating_function#Examples}): 

\begin{align}
m(r) & = \tilde{\mathbb{E}}(m(F)) \\
& = \tilde{\mathbb{E}}(\delta e^{-\gamma' F}) \\
& = \delta \tilde{\mathbb{E}}(e^{-\gamma' F}) \\
& \stackrel{\text{m.g.f. formula}}{=} \delta \exp \left( - \gamma' \tilde{\mu}_F + \frac{1}{2} \gamma' \tilde{\Sigma}_F \gamma \right) \\
& \stackrel{\text{Plug in } \eqref{posteriorMean} \eqref{posteriorCovariance}}{=} \delta \exp \left( - \gamma' (a + br) + \frac{1}{2} \gamma \tilde{\Sigma} \gamma \right) \\
& = \underbrace{\delta \exp \left( - \gamma' a + \frac{\gamma' \tilde{\Sigma} \gamma }{2} \right)}_{\text{Constant }C} \cdot \underbrace{e ^ {- \gamma' b r}}_{\text{Varying with }r}
\end{align}

Let's think about the constant and varying parts separately. 

\begin{itemize}
\item Constant part: this part deals with ``absolute level of discounting''. 
\item Varying part: the loadings are $\sum_j \gamma_j b_j$. It's unclearly whether high $\beta$ leads to higher discounting... in fact i would think it won't. 
\end{itemize}

What about the risk-aversion w.r.t. $r$? 

\begin{align}
m'(r) & = - \gamma' b \cdot m(r) \\
\Rightarrow - \frac{m'(r)}{m(r)} & = \gamma' b \\
& = \gamma' (\sigma_e^2 \Sigma_F^{-1} + \beta \beta')^{-1} \beta
\end{align}


\begin{itemize}
\item For each $j$ such that $\beta_j > 0$, clearly risk aversion is increasing in $\gamma_j$. 
\item I need to derive wrt $\beta$. The derivative should be positive... but the math does not seem to work out? 
\item Smaller $\sigma_e^2$ leads to larger risk aversion. 
\end{itemize}

\subsection{What about risk neutral distribution?}

Let's denote $m(r) = C e^{-\theta r}$, where $C$ is normalized appropriately, and $\theta = \gamma' b$. Let's realize that physical returns are normally distributed: 

\begin{equation}
R \sim \mathcal{N}(\mu_R = \beta' \mu_F, \sigma_R^2 = \beta' \Sigma_F \beta + \sigma_e^2)
\end{equation}

Let's multiple the densities. I'm ignoring the normalization constant. 

\begin{align}
q(r) & \propto m(r) p(r) \\
& \propto e^{-\theta r} \exp \left( -\frac{(r - \mu_R)^2}{2\sigma_R^2} \right) \\
& \stackrel{\text{Completing squares}}{=} \exp ( -\frac{1}{2 \sigma_R^2} \left[(r - (\mu_R - 2 \theta \sigma_R^2) )^2 + 4 \theta \mu_R \sigma_R^2 - 4 \theta^2 \sigma_R^4 \right] ) \\
& \propto \exp \left( \frac{-(r - \tilde{\mu}_R)^2 }{2} \right)
\end{align}

This this is a mean-shift from $\mu_R$ to $\tilde{\mu}_R$, where

\begin{align}
\tilde{\mu}_R & = \mu_R - 2 \theta \sigma_R^2 \\
& \stackrel{\text{plugging in}}{=} \beta' \mu_F - \underbrace{2 \gamma' \beta}_{\text{Price of risk}} \cdot \underbrace{(\beta' \Sigma_F \beta + \sigma_e^2)}_{\text{Risk}}
\end{align}

Higher risk loadings $\beta$ leads to higher price of risk for this stock. 

\section{How to actually do estimation?}

\begin{enumerate}
\item I think, $\Sigma_F, \mu_F, \beta, \sigma_e^2$ will have to be estimated in a first-stage. 
\item At each time $t$, do a normal fit on the RND of each stock's return. 
\item And then it comes to estimating $\gamma_t$, let's let risk-aversion by time-varying. Estimation is MLE with ex-post returns (requires some ``aggregate rational expectations'' assumption), because the reverse-transformed returns are iid uniform (Diebold has refs), so the likelihood function is easy to compute. 
\begin{itemize}
\item Perhaps we can allow $\Sigma_F, \mu_F$ to be somewhat time varying. I still have trouble thinking about how to make use of SPX's RND. It seems that it is just another return to throw into the MLE. 
\end{itemize}
\end{enumerate}

\subsection{Coding plan}

\begin{enumerate}
\item Need a routine to extract RNDs and fit a log-normal. This has to be robust. Each stock's parameters have to be slowly varying over time (need work)
\item The analytical transformations need to be coded up into utility functions (quick)
\item Estimate the beta-representations of each stock. Fix betas initially. 
\item Code up log likelihood estimation as a function. 
\item Do MLE... see if the parameters make sense over time. 
\end{enumerate}

\section{Some citations}


\end{document}
